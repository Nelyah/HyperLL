\documentclass{article}
\usepackage[utf8]{inputenc}
\usepackage{listings}
\usepackage{multicol}
\usepackage{amsmath}
\usepackage{color}


\title{HyperLogLog: Analysis and implementation of an improved algorithm}
\author{Dequeker Chloé, Ziat Ghiles}
\date{February 2015}

\begin{document}

\maketitle
\clearpage

\tableofcontents
\clearpage

\section{Introduction}

\section{cardinality estimation problem}
Finding the number of distinct elements in a data set with duplicates
is a well-known problem which applies in many fields.

The naive solution to this problem is to examine for each element of
the data stream its belonging to a data structure $\mathcal{D}$. If
$\mathcal{D}$ does not contain the element we add it to the data
structure. At the end of the process, the cardinality of the data
stream is equal to the size of $\mathcal{D}$.

This solution gives the exact answer but it is easy to see that it
scales very badly as the size of the data stream grows.

In order to resolve this problem, several alogorithms have been
proposed. These include LinearCounting and HyperLogLog which are the two
bases of the studied alogorithm.
\section{LinearCounting and HyperLogLog}
\subsection{LinearCounting}
\subsection{HyperLogLog}
The approach of the HyperLogLog algorithm to approximate the
cardinalities of a multiset is completely different. It is based on
randomization using a hash function for each element of the
multiset. It then focuses on the maximum of the number of leading
zeros in each hash values. it is legitimate to expect that the more
items there will be, the more this value will be high.
To improve the precision of this calculation, HyperLogLog uses the
stochastic averaging technique: Doing so, it splits the stream in
$m$ substreams, and perform the computation separately on each.

\section{HyperLogLog++}

\subsection{transition to 64 bits}

\subsection{Bias estimation and correction}
For a given configuration of the algorithm, the observed bias
is dependant on the cardinality estimated. From this observation, we
implement a correction method. 
\subsection{Memory optimization}
\subsubsection{Sparse representation}
\subsubsection{Dense representation}
\subsubsection{Varint encoding}
Since the temporary set used in the sparse representation is merged
with the list before it gets too large, performing a compression on it
is not as interesting then on the sorted list. We'll try to reduce the
memory usage of it by playing on two points:
\begin{itemize}
\item Using fixed-size integers as it is common practice in many
  langages may here result in a waste of memory space.
\item Since the manipulated list is sorted, we can take advantage of
  this information.
\end{itemize}

\section{Conclusion}


\end{document}
