\documentclass{article}
\usepackage[utf8]{inputenc}
\usepackage{listings}
\usepackage{multicol}
\usepackage{amsmath}
\usepackage{color}
\usepackage{graphicx}


\title{HyperLogLog: Analysis and implementation of an improved algorithm}
\author{Dequeker Chloé, Ziat Ghiles}
\date{February 2015}

\begin{document}

\maketitle
\clearpage

\tableofcontents
\clearpage

\section{Introduction}
In this paper, we present our implementation and analysis of a
caridnality esmation algorithm proposed by Stefan Heule, Marc
Nunkesser and Alexander Hall: \texttt{HyperLogLog++}. This algorithm
is itself an improvement of the \texttt{HyperLogLog} algorithm
proposed by Flajolet et. al.

\subsection{cardinality estimation problem}
Finding the number of distinct elements in a data set with duplicates
is a well-known problem which applies in many fields.

The naive solution to this problem is to examine for each element of
the data stream its belonging to a data structure $\mathcal{D}$. If
$\mathcal{D}$ does not contain the element we add it to the data
structure. At the end of the process, the cardinality of the data
stream is equal to the size of $\mathcal{D}$.

This solution gives the exact answer but it is easy to see that it
scales very badly as the size of the data stream grows.

In order to resolve this problem, several alogorithms have been
proposed. These include LinearCounting and HyperLogLog which are the two
bases of the studied alogorithm.
\section{LinearCounting and HyperLogLog}
\subsection{LinearCounting}
\subsection{HyperLogLog}
The approach of the HyperLogLog algorithm to approximate the
cardinalities of a multiset is completely different. It is based on
randomization using a hash function for each element of the
multiset. It then focuses on the maximum of the number of leading
zeros in each hash values. it is legitimate to expect that the more
items there will be, the more this value will be high.  To improve the
precision of this calculation, HyperLogLog uses the stochastic
averaging technique: Doing so, it splits the stream in $m$ substreams,
and perform the computation separately on each.

The result is then subjected to corrections:
\begin{itemize}
\item \emph{Small range correction :} As shown by simulation, for a
  cardinality smaller than $\frac{5}{2}$ of the number of substreams, non-linear
  distortions appear. For that range, LinearCounting is used.
\item \emph{Large range correction :} Due to the use of a 32 bit hash
  function, when the cardinality goes to $2^{32}$, the chances of hash
  collisions increases.
\end{itemize}

\section{HyperLogLog++}

\subsection{transition to 64 bits}
Using a 32-bits hash function restricts the area of efficiency of the
algorithm to the sets with less then $2^{32}$ distincts elements.
That's why an proposed improvement is to use a 64-bits hash function.
It does not significatively change the memory cost (it is
 onlys increased only by 1 bit per substream).

\subsection{Bias estimation and correction}
For a given configuration of the algorithm, the observed bias is only
dependant on the cardinality estimated. From this observation, we
implement a correction method: As shown in figure 1, the raw
estimation of HLL is distorted for small cardinalities. In order to
correct this error, we take measures of it for cardinalities between 0
and 100 000 (with a step of 500) and we store them into a file. From
now, the file will be loaded at the begining of the calculation. A
correction may then be calculated for the result using a linear
interpolation between the values registered and the raw estimations.

\begin{center}
\begin{figure}[h]
\includegraphics[scale=0.7]{img02.png}
\caption{Cardinality estimation for the corrected HLL}
\end{figure}
\end{center}


\subsection{Memory optimization}
Memory usage is an important factor in order to have a efficient
algorithm. In this algorithm, we can see that the size of the
different values we are using don't need to be of the standard size of
an int (which would be 4 bytes for a 32 bits integer). We need in fact
to keep two size of values, the first one being the index which
maximum value is $2^{P}$ with P the precision factor. That means any
index could be stocked on 14 bits. The second value is the number of
leading 0 of the hashed value which can't be over $64-P$ bits since we
work on a 64 bit version. The result is that the number of leading 0
will need 6 bits at most. The total size of those two values is then
of $6+P = 20$ bits. We will show in the next sections the different
kinds of compression we used during the implementation and in the
final state of the algorithm.

\subsubsection{Sparse representation}
This is the first type of compression, and is the one which should be
used when only a low number of index have been hashed. This
representation works by pairs (index, number of leading zero
(clz)). For a better understanding consider a bitmap, then separate it
by 20 bits blocks. Each of these block will be a pair (index,
value). The first P bits of the pair will represent the value of the
index, and the next 6 bits the value of clz. The total size (for P =
14) of the bitmap will then be the number of different index hashed
times 20 bits. We can then easily see why this representation is
particularly efficient for a low number of indexes and this is it's
strong perk. On the other hand, the more the number of different
indexes grow, the less efficient this representation becomes. \\ We
will then introduce the Dense representation, which becomes more
efficient when the number of index reaches a certain value we will be
talking about in the next section.

\subsubsection{Dense representation}
We will introduce in this section the second type of compression. As
we said earlier this representation if more efficient with a high
number of indexes. This picture this representation, we need here to
divid the bitmap by 6bits blocks. In this representation, when we go
through the first 6 bits of the bitmap, we will read the value of
index 0. The next 6 bits after that will be the value of index 1 and
so on. This representation allows us to represent the pair
(index,value) without writing the index. We can easily see this bitmap
will be of constant size since the value of the index is deducted from
the position of the 6-bit block in the bitmap. \\ Considering those
two representation, it is clear we need to start the algorithm using
the sparse representation, and then switch at some point to the dense
one. The limit where we want to switch from one to another is when the
sparse representation take more memory than the dense one. In the
implementation, it is then important to keep track of the bitmap's
size and switch to the dense representation whenever it is necessary.

\subsubsection{Varint encoding}
Since the temporary set used in the sparse representation is merged
with the list before it gets too large, performing a compression on it
is not as interesting then on the sorted list. We'll try to reduce the
memory usage of it by playing on two points:
\begin{itemize}
\item Using fixed-size integers as it is common practice in many
  langages may here result in a waste of memory space.
\item Since the manipulated list is sorted, we can take advantage of
  this information.
\end{itemize}


\section{Conclusion}

\end{document}
