\documentclass{article}
\usepackage[utf8]{inputenc}
\usepackage{listings}
\usepackage{multicol}
\usepackage{amsmath}
\usepackage{color}


\title{HyperLogLog: Analysis and implementation of an improved algorithm}
\author{Dequeker Chloé, Ziat Ghiles}
\date{February 2015}

\begin{document}

\maketitle
\clearpage

\tableofcontents
\clearpage

\section{Introduction}

\section{cardinality estimation problem}
Finding the number of distinct elements in a data set with duplicates
is a well-known problem which is applies in many fields.

The naive solution to this problem is to examine for each element of
the data stream its belonging to a data structure $\mathcal{D}$. If
$\mathcal{D}$ doesn't contain the element we add it to the data
structure. At the end of the process, the cardinality of the data
stream is equal to the size of $\mathcal{D}$.

This solution gives the exact answer but it easy to see that is scales
very badly as the size of the data stream grows.

\section{HyperLogLog}

\section{HyperLogLog++}
\subsection{transition to 64 bits}
\subsection{Bias estimation and correction}
\subsection{Memory optimization}
\subsubsection{Sparse representation}
\subsubsection{Dense representation}
\subsubsection{Varint encoding}
Since the temporary set used in the sparse representation is merged
with the list before it gets too large, it's not very relevant to
perform a compression on it. However, having no such a size limit for
the sorted list, we'll try to reduce its memory usage playing on two
points:
\begin{itemize}
\item Using fixed-size integers as it is common practice in many
  langages may here result in a waste of memory space.
\item Since the manipulated list is sorted, we can take advantage of
  this information.
\end{itemize}
\subsubsection{varint encoding}
\section{Conclusion}


\end{document}
